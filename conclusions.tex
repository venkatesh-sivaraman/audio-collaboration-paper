\section{Conclusions}

SimpleSpeech's intuitive design alleviates the pressure associated with the linearity of audio because users have the ability to easily fix errors after the fact.
Furthermore, we designed SimpleSpeech to clearly distinguish audio and text modalities, from the visual cues provided by the waveform to the quasi-modal interface for correcting transcription errors.
Students' use of these editing tools resulted greater feelings of comfort in producing comments than without SimpleSpeech functionality.
The true utility of this software, then, was to (at least partly) un-linearize audio, even making it more text-like.

Because studies of the linguistic and social characteristics of computer-mediated communication have been mostly limited to textual interactions, we also explored the formality and contextuality of AAC. 
Our finding that it was roughly in between spoken and written media is not discouraging \textit{per se}; however, the relatively formal characteristics of AAC must be taken into account before such a system is implemented in practice.
Nevertheless, we feel that the small-scale edits that users engaged in during this study are reassuring for potential applications of AAC.
Removing disfluencies and pauses allows users to feel comfortable with their recording while maintaining the spontaneity of thought in a spoken message.

The results of the qualitative study point to new directions for improving SimpleSpeech. 
For instance, on initial exposure to the application users initially tended to focus preferentially on the text instead of on the voice.
Slightly different visual layouts of the application, such as overlaying or juxtaposing the transcription on a more prominent waveform, could help users understand better that the text is a secondary tool.
Another possible feature could be automating certain edits, such as removing disfluencies and hesitations, to improve efficiency and edit quality even further.
Additionally, the findings in our quantitative study revealed promising trends concerning the benefits of AAC for online discussion, but may need a larger pool of test participants to attain statistical significance.

Our hope in developing SimpleSpeech is that asynchronous audio communication will gain greater usage in education and other collaborative settings. 
With the combination of ASR transcription for listeners and low-barrier editing tools for speakers, voice-based communication tools can engage students and improve the quality of collaboration on the Web.
