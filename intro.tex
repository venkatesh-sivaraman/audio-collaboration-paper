\section{Introduction}

Asynchronous audio communication (AAC) is rapidly becoming available to general users of technology. 
While text is still by far the most prevalent mode of communication on the Internet, social platforms such as WhatsApp, iMessage, and Facebook provide the option to record audio-only messages in addition to text, photo, and video. 
Audio is desirable in such situations because it allows users to deliver more expressive, nuanced messages.

AAC holds considerable potential for more specialized settings as well, namely collaboration for education and business. 
For instance, audio messages have been favorably used to disseminate feedback on assignments in online education courses \cite{ice,oomen}. 
On the other hand, voice communication is much less prevalent in business environments; the predominant manifestation of audio-only communication has been voicemail, a paradigm that is on the decline and generally regarded as ``onerous'' and ``laborious'' \cite{whittaker}. 
According to Grudin \cite{grudin}, the primary reason for this failure of voicemail is an imbalance of work between speaker and listener: message creators are afforded the convenience of rapid production, while listeners are often faced with a bevy of voice messages which must be processed essentially at the original rate of speech. 
Perhaps for this reason, email continues to predominate over voice messages in corporate communication despite the diminished individuality and, more critically, the higher risk of miscommunication in textual media \cite{byron}.

Despite the difficulties that Grudin points out, AAC is undoubtedly superior to text in certain contexts. 
For instance, audio messages could be used as annotations on virtual documents, enabling the user to consume others' comments aurally without distracting the user visually \cite{yoon}.
Another area of benefit is in online education, where voice communication has been shown to improve student-student and student-instructor engagement and a sense of the instructor's social presence \cite{oomen,tu}. 
However, we hypothesized that in applying AAC to online education, the burden on the listener may not be the only obstacle toward general use. 
Many younger students have trouble articulating their ideas vocally in physical classrooms, a detriment that would inevitably prevent these learners from participating in online oral discussions and render AAC useless.
The goal of software in such situations, then, is not only to improve consumers' efficiency but also to provide some tolerance for misspeaking and lack of fluency on the part of the producers.

Our solution is lightweight editing tools that students can use, based on automatic speech recognition (ASR) generated transcripts.
We designed and developed a user-friendly audio production tool, which we call SimpleSpeech, and applied it to the scenario of discussion on an online forum. 
The mental workload experienced by high school students and teachers recording voice messages was then studied, along with the linguistic features of the messages they produced.
