\section{Discussion}
In this study two forms of responding to pressure in a communication task were measured: the mental workload involved in completing the task and the degree of formality in the messages created in the task.
Using this information, we will evaluate the strengths and weaknesses of SimpleSpeech as a tool for enabling AAC as well as the viability of AAC in educational and collaborative contexts.

\subsection{Imbalance Between Speaker and Listener}
As Grudin notes, it is critical for collaborative software to spread the burden of usage equally on its constituent members. 
For instance, he cites email as a medium in which ``everyone generally shares the benefits and burdens equally'' \cite{grudin}. 
On the other hand, voice applications create inequality between speaker and listener since the former must expect that the latter will listen thoroughly and carefully to the message, a relatively slow task compared to reading.

However, the premise of SimpleSpeech is that the bias toward the speaker is reversed. 
ASR transcription can greatly facilitate the listener's task, as has already been demonstrated \cite{whittaker,vemuri}, bringing the workload down and closer to that of reading. 
Meanwhile, students who record messages could experience a \textit{greater} workload relative to writing because of the linearity of audio, which prevents them from correcting mistakes after the fact and thereby elevates the pressure to do well the first time.

SimpleSpeech was demonstrated to be a useful counterbalance in situations where the speaker's workload is elevated. 
In the qualitative study, some users noted the pressure ``to have organized thoughts'' and to ``sound composed more'' during recording, but that ``editing would make it nicer because you can go back and fix the mistakes'' ($P_2$).
Furthermore, the level of control was just right for most users: since they focused on deleting the disfluencies and pauses in their speech, the word-tokenized editor for the most part provided exactly the information needed to quickly delete undesirable sounds.
For the few users who did want to edit on a larger scale, the audio insertion feature was deemed helpful as well.

In the quantitative evaluation, we found strong evidence to support the use of SimpleSpeech, especially for students.
There was a significant decrease in task load on students when given the capability to edit, even in spite of the added time required to listen to the message and perform the editing. 
On the other hand, teachers did not find SimpleSpeech editing as useful, probably because they already perceived their recordings as being of acceptable quality. 
One teacher reasoned that he was ``already used to hearing [his] own voice'' from lecturing, a medium where statements cannot be retracted as easily as with SimpleSpeech.
However, many teachers did use the editing tools, even though their workload levels were not significantly different with or without this opportunity.
This would indicate that the editing tools are a valuable option for producers to have, but users should not be obligated to use them.

\subsection{Implications for Formality in AAC}
Our comparative analysis of formality using Heylighen et al.'s F-score \cite{heylighen} is unique in its application to mixed-media discourse.
While the corpora against which SimpleSpeech messages were tested were generated through either purely-verbal or purely-written means, users in this study were required to consider both the textual, semantic meaning of the transcript and the auditory, connotative meanings expressed by the voice.
As a result, our comparison of contextuality scores inherently lacks the auditory component that may be communicated through slang, dialect, or speaking rate.
We do not know of any one single measure that can capture these phenomena across different media.
However, our findings are useful for describing the formality of the textual message content, which is arguably the primary purpose of the communication as well as the aspect on which message consumers focused most.

The textual formality of the SimpleSpeech discussions was not significantly affected by the experimental conditions, indicating that for the most part, users are likely to adopt their own style for audio messages suitable for a discussion environment.
Nevertheless, it is critical to the success of general-purpose AAC that the formality of discussion be controlled to some extent, so that collaborators feel willing to participate.
For students this impetus toward quality is not as problematic, since they felt more relaxed rather than more stressed with editing functionality.
If discussion quality were driven up by artificial means, however, such as by grading students on the eloquence of their comments or evaluating employees on the basis of their online interactions, then individuals might gravitate toward socially ``safer'' modes of communication over which they feel more control (namely, text).
Proper acquaintance with audio editing capabilities is essential for AAC's survival under these pressures toward high-quality production.

To provide an example, discussion groups within an online course would be an ideal use of AAC using SimpleSpeech because students could send messages to a well-defined audience, thereby compensating for the additional formality imposed by the spatiotemporal distance between the participants. 
The editing tools would also drive students to produce better discussion input, increasing productivity and enhancing the learning experience.
On the other hand, enabling editing for personal communication, such as WhatsApp voice messages, would be detrimental to the desired informal speaking style of the platform.
Since the contextuality demanded by each situation is different, future audio-based collaboration platforms must consider the factors presented here and tailor their functionality accordingly.