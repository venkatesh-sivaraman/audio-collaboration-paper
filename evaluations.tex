\section{Qualitative Evaluation}

The interaction paradigm of SimpleSpeech was tested in a qualitative assessment to determine (1) the practicability of a lightweight text-based audio editor, (2) the effects of minor transcription errors on audio consumption and production, and (3) the implications of being able to edit audio in an asynchronous online discussion.

Participants were introduced to the functionality of the system, then given two untimed tasks. 
First, to simulate an asynchronous audio discussion, the test users were asked to read an audio comment left by the previous tester and create an audio response. 
Next, they received a different, textual prompt and created an audio comment which would be consumed by the next user. 
In both cases the user was asked to edit their recordings to be polished and clear.
The participants were interviewed at the end of the test over three general topics: (1) comparing the hybrid editor with conventional text editing, (2) comparing the discussion component with a text-based or face-to-face conversation, and (3) any user experience issues that occurred during the study. 
Afterwards, the interviews were transcribed, conversational elements filtered out, and the remaining sentences analyzed via two-step coding (open coding followed by flat coding). 

The sample for the study consisted of 9 test subjects (4 male, 5 female; henceforth denoted $P_1, P_2, \ldots, P_9$). 
All participants were native English speakers. 
Two individuals, $P_2$ and $P_3$, were professional media editors who provided technical feedback and a comparison to pure audio editing; the remainder were interns and high school students.

\subsection{Results}
The coding process for transcripts resulted in the following themes identified from the user feedback:

\emph{The text-based proxy provides sufficient control over elementary editing to supersede waveform manipulation.}
Most non-professional users felt SimpleSpeech gave them ``plenty of control'' over the editing process ($P_4,\,P_5,\,P_6,\,P_8$). 
The professional editors did note that most people in their field would not find this software adequate for their needs; but, as $P_2$ conceded, the intended market users ``don't have to play with the settings which is why they don't use a professional audio editor.''
Most participants characterized the editing experience as being a text-focused one, suggesting that the translation to text was in fact a useful proxy for editing audio. 
The text modality was described as ``more accessible, more doable'' than pure waveform editing, which could be ``scary for people who don't do video stuff'' ($P_3,\,P_7$). 

\emph{The primary use of lightweight voice editing is to make fine-grained rather than large-scale adjustments.}
The most commonly-used manipulation during the qualitative study was the removal of disfluencies ($P_1,\,P_2,\,P_4,\,P_5,\,P_7$), followed by space deletion ($P_2,\,P_3,\,P_5,\,P_6,\,P_8$). 
Only $P_1$ and $P_8$ edited large chunks of audio by deleting or rerecording, and $P_8$ reported doing so only to improve the smoothness of a smaller change in a sentence.
Perhaps because SimpleSpeech was presented as a tool to be briefly used to ``clean up'' recordings, participants focused on removing the ``embarrassing'' and ``awkward'' sounds ($P_1,\,P_5$).

\emph{Transcription is a helpful aid for listening to audio comments despite occasional errors.}
In many cases, the transcription proved to be an essential element of both the production and the consumption interfaces. 
To determine the effect of errors in the textual representation, the previous participants' comments were displayed to users with an unedited ASR transcript instead of the polished, human-edited one. 
Despite the occasional errors, users still found the ASR transcription to be helpful in allowing them to ``see all the points they were making instead of having to remember them'' ($P_4,\,P_6$). 
For some users, the transcription caused no problems in comprehension, while others experienced errors that required them to pay more attention to the audio ($P_8$). 
On the whole, ASR succeeded in ``getting the basic idea across'' ($P_3$) but could not stand alone without the original recording. 
In the editing scenario, most users agreed that the transcription functionality was ``overall quite accurate'' if they spoke clearly enough ($P_1,\,P_3,\,P_4,\,P_5,\,P_6,\,P_7,\,P_9$). 

\emph{The linearity of audio leads to a pressure to organize one's thoughts during recording.}
In some participants we also observed pressure associated with the audio production process. $P_4,\,P_7,$ and $P_9$ described a ``psychological sort of ... need to get it all out, and the fact that it won't necessarily be as organized there.'' 
Another tester, $P_5$, had ``a tendency to get like a blank slate'' in which he ``couldn't think of anything to say.'' 
The elevated mental task load that $P_5$ describes could be inherent in oral discussion; $P_9$ noted that ``[it] might just be the fact that I was recording,'' and in fact ``editing would make it nicer.'' 
Because this phenomenon was present despite the ability to edit and because it has not been explored in the literature to our knowledge, we decided to analyze the task load aspect of using SimpleSpeech in the quantitative study.

\emph{Awareness of the recipient and the editability of the audio drive up the quality of contributions.}
Interestingly, the participants' awareness of their audience and the ability to edit tended to drive up the quality of recordings.
Four users mentioned the formality of recordings using the software of their own volition or prompted by a question about pressure ($P_1,\,P_5,\,P_7,\,P_9$). 
$P_8$ described the situation as ``an expectation'' to edit, given that ``I know that I've had that opportunity and someone else would know that I had that opportunity.'' 
The speakers' inclination to consider their listeners is exemplified by $P_9$, when asked why she was motivated to edit her messages:
\begin{quote}
	Personally I'm editing to express myself a little more in a polished way when I'm writing.... especially if I know someone else is going to review it and be able to respond, I want to make sure I'm as clear as possible and as concise in a way that doesn’t really come across when I'm talking.
\end{quote}
Listening to another participant before initiating their own comment was likely a factor in determining the users' performance, since the exposure ``gave ... an understanding of how long of a comment, or what kind of direction people were trying to take the discussion'' ($P_9$). 
Editing contributed to the increased quality as well: ``Since you have the ability to edit things, it feels like you're talking to somebody who's prepared a point or a conversational view'' ($P_5$). 
We chose to explore this phenomenon quantitatively to determine if it was real or simply perceived by the speakers, and to what extent it was affected by the ability to edit.

\section{Quantitative Evaluation}
For our second experiment, we intended not just to assess the efficacy of SimpleSpeech in particular, but also to measure the usefulness of audio editing tools in general for educational discussions.
Participants in the study were asked to 

The quantitative study was conducted at a small public high school, with x students and x teachers.
